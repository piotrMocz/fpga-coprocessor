Trudno jednoznacznie wskazać metodykę wytwarzania oprogramowania którą posłużyliśmy się podczas realizacji projektu. Jako że nasze przedsięwzięcie miało bardzo szeroki zakres i obejmowało tematy takie jak parsowanie, typechecking, generacja kodu asemblera, realizacja procesora wektorowego i komunikacji UART spora część naszej pracy polegała na analizowaniu istniejących rozwiązań i planowaniu własnej architektury. Szczególnie na samym początku musieliśmy podjąć dużo decyzji co do syntaktyki i semantyki dostarczanego języka, mnemoników asemblera i szczegółów wewnętrznych procesora. Możnaby pokusić się o podział naszych prac niezależnie na 2 fazy lub 4 etapy. W początkowej fazie projektu rozważaliśmy czysty teoretycznie faktyczne działanie projektu. Idee te często się zmieniały i przechodziły drogę z rozwiązań najbardziej generycznych i uniwersalnych do wersji prostszych i mniej złożonych jednak o porównywalnej sile ekspresji. Rozwój i klaryfikacja koncepcji wymagały wdrażania się w wiele istniejących systemów oraz ich analizę pod kątem inspracji podczas szukania własnego rozwiązania. Drugą fazą w rozwoju projektu była implementacja - najpierw opracowanego języka wraz z całymi zawiłościami parsera, typecheckera potem generacja asemblera a następnie implementacja komunikacji UART i procesora wektorowego. Sam procesor - ściśle powiązany z językiem i asemblerem implementowany był przyrostowo. Rozpoczynając od najprostszej wersji rozszeraliśmy ją o kolejne funkcjonalności. Także podczas tego etapu wiele koncepcji zmieniało się lub przechodziło niezbędne poprawki. Podział na etapy w realizacji projektu wyznaczają z pewnością kolejne spotkania z opiekunem pracy podczas których pojawiały się nowe propozycje rozwiązań konkretnych problemów oraz weryfikowane były zaproponowany przez nas pomysły.