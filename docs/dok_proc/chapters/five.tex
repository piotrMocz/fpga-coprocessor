Realizowany przez nas sprzętowo-programowy system przetwarzania danych wektorowych, oparty na układzie FPGA został zrealizowany na czas. Sama koncepcja wewnętrznego działania i implementacji zmieniała się wielokrotnie w trakcie realizacji(przejście na architekture stosową, zmiany w binaryzacji mnemoników). Wszystkie wymagania funkcjonalne zostały spełnione a ich implementacja zakończyła się sukcesem. Mimo dużej złożoności problemu i rozpiętości podjętych zagadnień projekt posiada bardzo jasną, poukładaną strukture. Umożliwia to jego dalszy rozwój i skalowanie w łatwy i dogodny sposób.\\Propozycje rozszerzeń:
\begin{itemize}
  \item Dodanie do języka nowych rodzajów pętli
  \item Obsługa funkcji w języku
  \item Implementacja funkcji wejścia-wyjścia - np. czytania z pliku
  \item Obsługa liczb zmiennoprzecinkowych przez procesor
  \item Uruchomienie procesora na bardziej złożonych płytkach FPGA - wydajniejszych i z większą ilością bramek logicznych
\end{itemize}