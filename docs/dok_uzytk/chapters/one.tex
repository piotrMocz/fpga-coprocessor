\section{Instalacja systemu i niezbędnych składników}
Podstawowym składnikiem potrzebnym do skompilowania haskellowej części projektu jest platforma Haskella. W skład jej wchodzą:
\begin{description}
  \item[Glasgow Haskell Compiler] \hfill \\
  otwarty kompilator i interaktywne środowisko funkcjonalnego języka Haksell 
  \item[Cabal build system] \hfill \\
  architektura do budowania bibliotek i aplikacji dedykowana dla Haskella
  \item[35 pakietów] \hfill \\
  Podstawowe i najczęściej używane moduły haskella
  \item[Narzędzia do profilowania i analizowania kodu] \hfill \\
  Interaktywne środowisko programowania, parsery i lexery haskellowego kodu, środowisko testowe
\end{description}

Platforme Haskella znaleźć można na oficjalnej stronie języka w zakładce Downloads(https://www.haskell.org/downloads).
Po zainstalowaniu platformy należy wywołać komendę \texttt{cabal install} w folderze \texttt{fpga-coprocessor/lang}. Komenda skompiluje kod źródłowy kompilatora i skopiuje go wraz z zależnościami do folderu \texttt{dist}.

Kolejnym wymaganym składnikiem jest środowisko Quartus II Web Edition potrzebne do kompilacji i uruchomienia projektu na zestawie startowym z układem fpga. Pobrać można je z centrum pobierania na stronie Altery (http://dl.altera.com).
Po otworzeniu projektu należy go skompilować klikając \textbf{Start Compilation} a następnie przesłać do urządzenia(\textbf{Programmer}\textgreater \textbf{Start}).

Ostatnim wymaganym programem jest kompilator języka C - GCC, the GNU Compiler Collection potrzebny do kompilacji programu wysyłającego plik binarny na zestaw uruchomieniowy i odbierającego wynik. Program skompilować można na przykład poleceniem 
\texttt{gcc send\_bin.c -o send\_bin}
\clearpage
\section{Konfiguracja sprzętu}
Do uruchomienia projektu potrzebny jest zestaw startowy z układem FPGA. Projekt rozwijany i testowany był na układzie \textbf{Terasic DE0-Nano z rodziny Cyclone IV} firmy Altera.

\begin{figure}[h]
\begin{center}
\includegraphics[scale=0.75]{images/de0-nano}
\caption{Zestaw startowy Terasic DE0-Nano Board}
\end{center}
\end{figure}

Komunikacja komputera PC z zestawem FPGA odbywa się za pomocą układu UART w związku z tym do przesłania pliku potrzebny jest \textbf{Konwerter USB-UART}. W projekcie wykorzytaliśmy Konwerter USB-UART PL2303 umożliwiający komunikacje pomiędzy interfejsami szeregowymi USB i UART widoczny w systemie jako wirtualny port COM.
\begin{figure}
\centering
\includegraphics[scale=0.1]{images/usb-uart}
\caption{USB-UART PL2303 - wtyk USB}
\end{figure}
Pin GND konwertera należy połączyć z Pinem GND zestawu startowego(np. PIN12 JP1 lub PIN12 JP2). Pin TXD konwertera łączymy z pinem GPIO 11 natomiast Pin RXD z pinem GPIO 9.

Zestaw startowy należy podłączyć do komputera również poprzez dołączony kabel USB co umożliwi wgrywanie oprogramowania poprzez interfejs USB-BLASTER. Złącza RX i TX zaleca się oddzielić opornikiem o wartości 1K\ohm\space jednak z własnego doświadczenia możemy dodać, że nie jest to konieczne.
\begin{figure}
\centering
\includegraphics[scale=0.1]{images/full_connected}
\caption{Podpięty cały zestaw}
\end{figure}
\section{Krótki kurs języka}
\subsection{Znaki białe}
W zaproponowanym przez naz języku białe znaki (odstęp, nowa linia, wcięcie) są przez parser ignorowane.
Rozwiązanie takie daje możliwość formatowania kodu w sposób najbardziej elastyczny, zależny od potrzeb użytkownika. To samo wyrażenie można zapisać w wielu liniach na sposób możliwie czytelny lub jednej by otrzymać zwięzły, niewielki wizualnie kod.
Dla przykładu dane poniżej 2 formatowania skutkują w równoważnym kodzie:
\begin{lstlisting}[frame=single]
x:Int = if a
        then b = a * 2
             b + a
        else b = (a + 1) * 2
             b - 1
        end
\end{lstlisting}
\begin{lstlisting}[frame=single]
x:Int = if a then b=a*2 b+a else b=(a+1)*2 b-1 end
\end{lstlisting}

\subsection{Komentarze}
Dane są 2 rodzaje komentarzy - liniowe \texttt{\#} oraz blokowe \texttt{\string{\# ... \#\string}}
\begin{lstlisting}[frame=single]
# komentarz liniowy
{#
komentarz
blokowy
...
#}
\end{lstlisting}

\subsection{Literały}
W aktualnej fazie rozwoju projektu język dostarcza 2 rodzaje literałów - liczby całkowite oraz tablice liczb całkowitych:
\begin{lstlisting}[frame=single]
1024
[1, 2, 4, 8, 16, 32, 64, 128]
\end{lstlisting}

\subsection{Zmienne, deklaracje zmiennych i przypisanie do zmiennej}
Każde wyrażenie w języku ma jeden z dwóch typów - \textbf{Vector} lub \textbf{Scalar}. Dodatkowo typ Vector parametryzowany jest jedną liczbą naturalną - długością wektora.
Każda zmienna przed użyciem musi zostać zadeklarowana. Nie można zadeklarować zmiennej nie inicjalizując jej jednocześnie. Mechanizm ten chronić ma przed przypadkowymi błędami częstymi w innych językach. \\
Przykłady deklaracji:
\begin{lstlisting}[frame=single]
# deklaracje zmiennych skalarnych
a:Int = 4
b:Int = 2*a

# deklaracje wektorow
x:IntVector[4] = [0,0,1,1]
y:IntVector[5] = [1,2,3,4,5]

# przypisania
a = 4
a = [1,2,3] # zle! a ma typ Scalar

z:Int #zle! przy deklaracji trzeba zainicjowac zmienna

\end{lstlisting}


\subsection{Operatory}
Język posiada 7 operacji wspierających obliczenia wektorowe:
\begin{description}
  \item[+] Dodawanie - operacja na 2 skalarach, lub 2 wektorach - wtedy element po elemencie
  \item[*] Mnożenie - operacja na 2 skalarach, lub 2 wektorach - wtedy element po elemencie
  \item[-] Odejmowanie - operacja na 2 skalarach, lub 2 wektorach - wtedy element po elemencie
  \item[/] Dzielenie - operacja na 2 skalarach, lub 2 wektorach - wtedy element po elemencie
  \item[?] Iloczyn skalarny - operacja na 2 równej długości wektorach - zwraca skalar
  \item[\%] Modulo - operacja na 2 skalarach, lub 2 wektorach - wtedy element po elemencie
  \item[rot90] - rotacja 4 punktów w przestrzeni dwuwymiarowej o 90 stopni względem punktu (256, 256)
\end{description}

Przykłady działania operacji:
\begin{lstlisting}[frame=single]
[2,3,4,5,6] + [1,2,3,4,5] # =[3,5,7,9,11]
[1,2,4,8] - [1, 1, 2, 4] # =[0,1,2,4]
[1,2,3,4,5,6] * [2,2,2,2,2,2] # =[2,4,6,8,10,12]
[8,6,4] / [2,2,2] # =[4,3,2]
[1,1,1,1] ? [2,2,2,2] # =8
[64, 7, 24] % [8, 3, 16] # =[8, 3, 16]
rot90 [10,10,10,10,10,10,10,10] 
    # =[10, 246, 10, 246, 10, 246, 10, 246]
\end{lstlisting}

\subsection{Instrukcje warunkowe}
Instrukcja warunkowa ma postać:
\begin{center}
\textbf{if} \textit{expression} \textbf{then} \textit{expression\_block} \textbf{else} \textit{expression\_block} \textbf{end}
\end{center}
Wartością zwracaną przez instrukcje jest ostatnie wyrażenie w pierwszym lub drugim bloku. Typem instrukcji jest typ ostatniego wyrażenia w obu blokach. Wynika z tego, że wartości zwracane przez oba bloki muszą mieć ten sam typ, by kod mógł zostać poprawnie otypowany. Blok pierwszy wykonywany jest jeśli wyrażenie warunkowe było różne od zera lub nie było wektorem zer, natomiast drugi w odwrotnym przypadku.
Przykład użycia:
\begin{lstlisting}[frame=single]
a:Int = 0
b:IntVector[4] = [0,1,2,3]
c:IntVector[4] = if a then b else b + [1,1,1,1] end
# c = [1,2,3,4]

d:IntVector[4] = if a+1 then b else b ? [1,2,1,2]
# niezgodne typy: w bloku pierwszym wektor, w drugim skalar 
\end{lstlisting}

\subsection{Pętla loop}
Składnia pętli loop:
\begin{center}
\textbf{loop(} \textit{expression} \textbf{):} \textit{expression\_block} \textbf{end}
\end{center}
Pętla loop jako jedyna formuła w języku nie zwraca żadnego typu i służy do repetywnego wykonywania kodu podanego w ciele pętli. Jeśli ilość iteracji jest skalarem \textit{n} pętla wykonywana jest \textit{n}-krotnie. Jeśli wyrażenie wewnętrz okrągłych nawiasów jest wektorem to ilość powtórzeń instrukcji jest ostatnim elementem wektora.\\
Przykład:
\begin{lstlisting}[frame=single]
term2:IntVector[8] = [1,1,1,1,1,1,1,1]
result:IntVector[8] = [0,0,0,0,0,0,0,0]

loop(254):
  result = result + term2
end
result
#expecting [254, 254, 254, 254, 254, 254, 254, 254] 
\end{lstlisting}

\section{Kompilacja}
\subsection{Użycie kompilatora}
Do kompilacji należy użyć programu \textit{compilator} napisanego w haskellu i wygenerowanego w folderze \textit{/dist}(patrz 1.1). Kompilowany plik powinien się znajdować w aktualnym folderze i mieć nazwę \textit{example}.
Podczas kompilacji na ekran wypisane zostaną informacje takie jak informacje o kolejnych etapach kompilacji, drzewo AST programu, wygenerowany kod ASM oraz użyte w programie stałe.
\begin{figure}[!h]
\centering
\includegraphics[scale=0.75]{images/kompilator}
\caption{Kompilacja programu}
\end{figure}
\clearpage
\subsection{Możliwe błędy}
Proces kompilacji może zostać przerwany przez następujące błędy:
\begin{description}
  \item[Address of JUMP too big]\hfill \\
       Błąd wewnętrzny kompilatora - rozmiar kodu jest na tyle duży, że numer pewnej linijki nie zmieści się w rejestrze na FPGA.
  \item[Address of Push, Load or Store too big] The second item\hfill \\
       Kolejny błąd kompilatora - duży kod powoduje, że adres binarny zmiennej jest większy niż przewiduje architektura procesora.
  \item[No more address space for new variables] \hfill \\
       Błąd wynikający z dobranych rozmiarów pamięci na FPGA - nie ma już miejsca na kolejną zmienną w pamięci.
  \item[Assigning to non-existing variable]
       Próba przypisania do niezadeklarowanej wartości. Na przykład, w podanym poniżej programie użytkownik przypisuje wartość do niezadeklarowanej zmiennej a.
\begin{lstlisting}[frame=single]
a:Int = 2
\end{lstlisting}
  \item[Wrong type of X]\hfill \\
      Niezgodność typów podczas deklaracji zmiennej. Przykład - do zmiennej typu \textbf{Int} przypisanie wartości       \textbf{IntVector[5]}:
\begin{lstlisting}[frame=single]
a:Int = [1,2,3,4,5]
\end{lstlisting}  
  \item[X and Y differ] \hfill \\
      Częsty błąd wielu operacji binarnych. Próba np. przemnożenia 2 wektorów o różnych rozmiarach:
\begin{lstlisting}[frame=single]
[1,2,3,4,5] * [1,2]
\end{lstlisting}       
  \item[Rotation vector needs exactly 8 elements] \hfill \\
      Błąd operacji \textbf{rot90}. Wymaga ona współrzędnych 4 punktów w 2 wymiarach podczas gdy użytkownik podaje inną ilość.
\begin{lstlisting}[frame=single]
rot90 [100, 100, 230] # error!
\end{lstlisting}       
  \item[Empty statement block in if structure] \hfill \\
      Użytkownik nie wpisał żadnego wyrażenia do któregoś z bloków pętli.
\begin{lstlisting}[frame=single]
x:Int = if 0
        then # error!
        else
            2 * 5
        end
\end{lstlisting}        
  \item[Then and else returns different structs] \hfill \\
       Typ bloku \textbf{then} i \textbf{else} powinien być taki sam. W innym wypadku system sprawdzania typów nie zezwoli na kompilacje.
\begin{lstlisting}[frame=single]
x:Int = if 0
        then 
           [1,2,3]  # Vector[3]
        else
            2 * 5   # Scalar
        end
\end{lstlisting}   
  \item[Variable X undeclared] \hfill \\
      Użyto zmiennej która nie została zadeklarowana.
\begin{lstlisting}[frame=single]
x:Int = a * 2 # error!
\end{lstlisting}   
  \item[Loops dont return values] \hfill \\
     Użytkownik próbował przypisać pętle do zmiennej.
\begin{lstlisting}[frame=single]
y:Int = 0
x:Int = loop(5): # error!
           y = y + 1
        end  
\end{lstlisting} 
  \item[Variable X already declared] \hfill \\
      Podwójna deklaracja zmiennej.
      \begin{lstlisting}[frame=single]
y:Int = 0
y:Int = 1
\end{lstlisting} 
  \item[Cannot assign X to Y of type]\hfill \\
  Niezgodność typów przy przypisywaniu do zmiennej.
  \begin{lstlisting}[frame=single]
x:IntVector[5] = [1,2,3,4,5] ? [1,2,3,4,5]
\end{lstlisting} 
\end{description}

\section{Wysyłanie programu i odbieranie wyników}
Skompilowany kod zapisywany jest w binarnej reprezentacji do pliku \textit{binarka}. Jest on gotowy do wysłania na płytkę FPGA. Służyć do tego może poprzednio skompilowany program \textit{send\_bin}.
\begin{center}
sendbin binarka
\end{center} 
Po wysłaniu programu na zestaw startowy zostanie on wykonany, a następnie na ekranie wyświetli się wynik programu.
\begin{figure}[!h]
\centering
\includegraphics[scale=0.75]{images/exec1}
\caption{Kod źródłowy}
\end{figure}
\begin{figure}[!h]
\centering
\includegraphics[scale=0.75]{images/exec2}
\caption{Kompilacja kodu}
\end{figure}
\begin{figure}
\centering
\includegraphics[scale=0.75]{images/exec3}
\caption{Wysyłanie kodu na urządzenie}
\end{figure}
\begin{figure}
\centering
\includegraphics[scale=0.75]{images/exec4}
\caption{Odbieranie wyniku}
\end{figure}