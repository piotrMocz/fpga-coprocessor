Jak ukazano w powyższym dokumencie, proces pisania i uruchamiania programów na zaimplementowanym procesorze jest stosunkowo prosty i nie wymaga znajomości asemblera, wiedzy na temat zaawansowanej technologii FPGA ani teorii kompilacji. Równocześnie sam procesor wspierając operacje wektorowe daje możliwość wykonania wielu różnorodnych obliczeń (w tym przykładowych, podanych w poprzednim rozdziale) w bardzo wydajny sposób. Na szczególną uwagę zasługuje łatwość z jaką daje się dzięki zaimplementowanemu rozwiązaniu tłumaczyć wysokopoziomowe programy operujące w abstrakcyjnej przestrzeni na kod asemblera który potem uruchamia się na wpełni wyspecjalizowanym ku takim zadaniom procesorze. 