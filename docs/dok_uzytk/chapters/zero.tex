W niniejszym dokumencie dostarczamy użytkownikowi instrukcje krop po kroku opisującą korzystanie ze zrealizowanego przez nas produktu i dajemy konkretne wskazówki co do jego użytkowania. Zamierzeniem naszego projektu jest dostarczenie odbiorcy: kompilatora, narzędzi do przesyłania programu na FPGA oraz w pełni funkcjonalnego procesora uruchomionego na FPGA. Będąc jednak świadomi, że dociekliwy użytkownik może próbować samemu skompilować źródła naszego projektu lub nawet dokonywać w nim zmian w celu dalszego rozwoju w tym dokumencie dostarczamy również informacji na temat tego jak nasz kod źródłowy skompilować. W pierwszym rozdziale opisujemy również jak podłączyć do komputera zestaw startowy i jak przesłać na niego program. Opisujemy dokładnie składnie zaimplementowanego przez nas języka wraz z wieloma przykładowymi kodami źródłowymi. Podajemy wszystkie możliwe błędy które zdarzyć się mogą podczas kompilacji programu. W drugim rozdziale proponujemy na konkretnych przykładach do jakich zastosowań użytkownik może wykorzystać zrealizowany procesor wektorowy.